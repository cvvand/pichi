\documentclass[paper=a4, oneside]{memoir}

\usepackage[T1]{fontenc}
\usepackage[english]{babel}
\usepackage{amsmath,amsfonts}
\usepackage{tensor}
\usepackage{graphicx}

\usepackage{listings}
\usepackage{xcolor}

\chapterstyle{thatcher}

\usepackage{chngcntr}
\counterwithout{table}{chapter}

\graphicspath{{./diagrams/}}


\newcommand{\tens}[2]{\tensor{#1}{#2}}
\newcommand{\packagename}{pichi}


\title{Using \packagename}
\author{Christian Walther Andersen\thanks{cwandersen@imada.sdu.dk}}
%\date{}

\begin{document}

\lstdefinestyle{cc++}{
	belowcaptionskip=1\baselineskip,
	breaklines=true,
	frame=L,
	xleftmargin=\parindent,
	language=C++,
	showstringspaces=false,
	basicstyle=\ttfamily,
	keywordstyle=\bfseries\color{green!40!black},
	commentstyle=\itshape\color{purple!40!black},
	identifierstyle=\color{black},
	stringstyle=\color{orange},
}

\lstset{style=cc++}

\maketitle % Print the title

\chapter*{The basics}

This chapter describes how to use \packagename to do simple tensor 
contractions. There are two central classes used for this purpose. The Tensor 
class (\texttt{Tensor.h} and \texttt{Tensor.cc}) and the Contraction class 
(\texttt{Contraction.h} and \texttt{Contraction.cc}). The Tensor class 
essentially manages the storage of the data for each individual tensor, while 
the Contraction class serves as a container for the tensors as well as 


\begin{lstlisting}
Tensor t( 3 , 64 );
\end{lstlisting}















\end{document}